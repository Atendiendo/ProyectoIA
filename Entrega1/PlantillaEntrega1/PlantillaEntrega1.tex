\documentclass[letter, 10pt]{article}
\usepackage[utf8]{inputenc}
\usepackage[T1]{fontenc}
\usepackage[spanish]{babel}
\usepackage{amsfonts}
\usepackage{amsmath}
\usepackage[dvips]{graphicx}
\usepackage{url}
\usepackage[top=3cm,bottom=3cm,left=3.5cm,right=3.5cm,footskip=1.5cm,headheight=1.5cm,headsep=.5cm,textheight=3cm]{geometry}



\begin{document}
\title{Inteligencia Artificial \\ \begin{Large}Estado del Arte: Enhanced Profitable Tour Problem\end{Large}}
\author{Miguel Huerta Flores}
\date{\today}
\maketitle


%--------------------No borrar esta secci\'on--------------------------------%
\section*{Evaluación}

\begin{tabular}{ll}
Resumen (5\%): & \underline{\hspace{2cm}} \\
Introducci\'on (5\%):  & \underline{\hspace{2cm}} \\
Definici\'on del Problema (10\%):  & \underline{\hspace{2cm}} \\
Estado del Arte (35\%):  & \underline{\hspace{2cm}} \\
Modelo Matem\'atico (20\%): &  \underline{\hspace{2cm}}\\
Conclusiones (20\%): &  \underline{\hspace{2cm}}\\
Bibliograf\'ia (5\%): & \underline{\hspace{2cm}}\\
 &  \\
\textbf{Nota Final (100\%)}:   & \underline{\hspace{2cm}}
\end{tabular}
%---------------------------------------------------------------------------%
\vspace{2cm}


\begin{abstract}
El presente informe analiza el Enhanced Profitable Tour Problem (EPTP), una generalización del Traveling Salesman Problem en la que no
es obligatorio visitar todos los nodos y cada nodo y arco posee una valorización propia. Se describen sus restricciones operativas,
como ventanas de tiempo y tiempos de servicio, y se presentan las variantes más relevantes. Se realiza una revisión del estado del
arte, comparando métodos exactos, heurísticos, metaheurísticos, híbridos y técnicas emergentes aplicadas a problemas relacionados.
Finalmente, se propone un modelo matemático que formaliza el problema y se discuten las metodologías más efectivas para su resolución,
destacando los enfoques híbridos y metaheurísticos por su eficiencia y adaptabilidad a escenarios reales.
\end{abstract}


\section{Introducción}
En los últimos años han surgido diversas empresas y aplicaciones que se centran en definir la ruta más óptima (Reparto urbano,
rutas turísticas, etc.) para maximizar la
rentabilidad del negocio. El Enhanced Profitable Tour Problem (EPTP) es una generalización
del Traveling Salesman Problem (TSP): no es necesario visitar todos los nodos, cada nodo tiene una valoración propia según el usuario
y se busca maximizar el beneficio neto del recorrido teniendo en cuenta tanto las ganancias por visitar nodos como los arcos.

Este documento se divide en cuatro partes:
\begin{itemize}
  \item \textbf{Definición del problema:} Se explica con detalle en qué consiste el EPTP, cuáles son sus variables, restricciones y
  objetivo(s). Además se presentan problemas relacionados y las variantes más conocidas.
  \item \textbf{Estado del arte:} Se detallan los estudios previos relevantes, comparando métodos exactos, heurísticos,
  metaheurísticos, hibridos (completos e incompletos) y técnicas emergentes aplicados a variantes cercanas.
  \item \textbf{Modelo matemático:} Se describe el modelo matemático propuesto/analizado, incluida la notación y las ecuaciones
  principales.
  \item \textbf{Conclusiones:} Se exponen las conclusiones principales extraídas del análisis y se sugieren líneas futuras de trabajo.
\end{itemize}

El propósito de este documento es profundizar en el problema EPTP, revisar trabajos relacionados y analizarlos
para asi intentar determinar la mejor forma de resolver este problema.


\section{Definición del Problema}
Sea un grafo dirigido $G = (V, A)$, donde $V$ representa un conjunto de $n$ nodos y $A$ un conjunto de $m$ arcos. El objetivo del
problema consiste en seleccionar un subconjunto de nodos de manera que se maximice la valorización neta del recorrido. Tanto los
nodos como los arcos tienen asociadas una valorización y un tiempo de uso —tiempo de servicio en los nodos y tiempo de viaje en los
arcos— que dependen del usuario que realiza el tour.

En esta versión del problema, además, cada nodo cuenta con una ventana de tiempo $[e_i, l_i]$, que indica el intervalo en el cual
dicho nodo puede ser visitado.

Dado que el problema es una generalización del TSP, esto implica que es NP-Hard. Por lo tanto, 
en la práctica resulta inviable resolver instancias de tamaño medio o grande mediante métodos exactos
en tiempos computacionales razonables. Por esta razón, la literatura reciente se ha enfocado en el desarrollo 
de enfoques aproximados. Estos métodos buscan obtener soluciones
de alta calidad en tiempos reducidos, permitiendo abordar aplicaciones reales en ámbitos como la logística, 
la planificación de rutas de servicio y la gestión del transporte.

\textbf{Restricciones del problema}:
\begin{itemize}
    \item El tour comienza y termina en el nodo 1
    \item El tour consiste en un subconjunto de nodos
    \item Cada nodo del tour es visitado a lo más una vez
    \item Todos los usuarios comienzan en un tiempo inicial $t = 0$.
    \item La duración total del recorrido no debe exceder el tiempo total de cada usuario
    \item Cada nodo $i$ sólo puede visitarse dentro de su ventana de tiempo [$e_i$, $l_i$].
    \item Si se arriba antes de $e_i$, se espera; si se arriba después de $l_i$, el nodo no puede visitarse.
\end{itemize}

\textbf{Variantes mas conocidas}:
\begin{itemize}
  \item \textbf{Profitable Tour Problem Clásico}: Se combinan beneficios por visitar nodos y los costos por visitar nodos. Maximizando el
  beneficio neto. \cite{DASARI2021100897}
  \item \textbf{Orienteering Problem (OP)}: Se busca pasar por nodos (no necesariamente todos) con ``premios'' para maximizar ganancias, 
  dentro de un limite de tiempo. \cite{5668224}
  \item \textbf{Team Orienteering Problem (TOP)}: Se busca encontrar rutas para multiples vehiculos en un grafo, tal que,
  el profit de la suma de los nodos sea maximizado, sujeto al costo de que conlleva cada vehiculo. \cite{9155343}
  \item \textbf{Prize-Collecting TSP (PCTSP)}: Generalizacion del TSP. Al visitar un nodo, se recolecta un ``premio'' y por cada
  nodo no visitado, se penaliza. El objetivo es minimizar la suma de los costos de viaje y las penalizaciones, incluyendo en
  el tour, un numero suficiente de nodos para recolectar un ``premio'' minimo. \cite{1587725}
\end{itemize}

\section{Estado del Arte}

\subsection{Origen histórico del problema}
El problema de enrutamiento tiene su base en el conocido Traveling Salesman Problem (TSP).
La formulación clásica del TSP es rígida, ya que impone la visita obligatoria a todos los nodos de una instancia. Si
bien esto es válido para muchos contextos teóricos, en la práctica logística esta premisa puede llevar a soluciones ineficientes y
poco rentables. Surgen, por ejemplo, situaciones en las que el costo de viajar hasta un cliente remoto supera el beneficio de atenderlo,
o donde sus restrictivas ventanas de tiempo hacen inviable su inclusión en una ruta cohesiva.

Esta limitación de los modelos clásicos motivó el desarrollo de variantes más flexibles y realistas. Una de las más relevantes es
el Profitable Tour Problem (PTP), que introduce un cambio de paradigma fundamental: no es obligatorio visitar todos los clientes.
En su lugar, el PTP busca maximizar la utilidad neta de la ruta, definida como la diferencia entre la suma de los beneficios (o premios)
obtenidos al visitar clientes y el costo total del recorrido.

El Enhanced Profitable Tour Problem (EPTP) surge entonces como una evolución natural del PTP, incorporando restricciones operativas
críticas del mundo real (como ventanas de tiempo y tiempos de servicio) que condicionan la viabilidad y
rentabilidad del tour. Así, el EPTP no solo decide de forma óptima a qué clientes visitar, sino también en qué orden hacerlo,
respetando las restricciones que modelan el contexto operativo.

\subsection{Enfoques y técnicas utilizadas}

\subsubsection{Métodos exactos}
Este enfoque es presentado por \cite{LERAROMERO2021879}, quienes abordan el Problema de
time-dependent profitable tour problem with resource constraints (TDPTPRC). En este trabajo, los autores
desarrollan un algoritmo de Branch and Cut específicamente diseñado para este problema, que generaliza el Profitable Tour
Problem (PTP) al incorporar tiempos de viaje variables para modelar la congestión vial. Proponen una formulación de
programación lineal entera mixta (MILP) que, explotando la estructura de la función de tiempo de viaje,
reducen significativamente el número de variables y restricciones.

\subsubsection{Heurísticas}
En este contexto, \cite{1587725} propone dos algoritmos híbridos para resolver el Prize collecting travelling salesman problem
(PCTSP): el Evolutionary Clustering Search (ECS) y una adaptacion de este llamado *CS. Estos métodos se basan en una estrategia de
clustering de búsqueda para detectar áreas prometedoras del espacio de soluciones y luego aplicar métodos de búsqueda local para
refinarlas.

Por otro lado, \cite{9155343} aborda el Team orienteering problem (TOP), proponiendo
algoritmos de aproximación con ratios teóricos garantizados. Los autores desarrollan un algoritmo greedy con un
ratio de aproximación de $(1-(1/e)^{1/(2 + \epsilon)})$ para el caso general, y un algoritmo mejorado para el
caso especial donde todos los vehículos son del mismo tipo. Las evaluaciones experimentales muestran que sus algoritmos
superan en un 12.5\% a 17.5\% a los enfoques existentes en términos de profit obtenido.

Y por último, \cite{DASARI2021100897} se centra en el Profitable Tour Problem (PTP), proponiendo tres métodos: multi-start
hyper-heuristic (MSHH), multi-start iterated local search (MS-ILS) and multi-start general variable neighborhood search (MS-GVNS).
La hyper-heurística MSHH utiliza ocho heurísticas de bajo nivel (como adición, eliminación, intercambio y 2-opt) y dos mecanismos
de selección: aleatorio (MSHH\_RAND) y greedy (MSHH\_GREEDY). Los experimentos en 77 instancias basadas en TSPLIB muestran que
MSHH\_RAND y MSHH\_GREEDY superan consistentemente a MS-ILS y MS-GVNS en la mayoría de los casos, tanto en calidad de solución best
como average, con MSHH\_RAND siendo el método más efectivo en general, especialmente bajo criterios de tiempo corto.

\subsubsection{Metaheurísticas}
Por el lado de las Metaheurísticas, \cite{5668224} propone un algoritmo de Modified Variable Neighborhood Search (MVNS) para resolver el
Orienteering Problem (OP). A diferencia de los métodos clásicos de Variable Neighborhood Search (VNS), el MVNS divide el espacio de
búsqueda en sub-regiones según los niveles de restricción. Durante la búsqueda, selecciona una solución base de una lista de candidatos
que contiene una solución por cada nivel de restricción, y las soluciones vecinas encontradas se comparan y actualizan globalmente
en dicha lista. El algoritmo incorpora dos tipos de procedimientos de vecindad: path improvement (intercambio, 2-opt e inserción,
tanto intra-ruta como inter-rutas) y score improvement (reemplazo e inserción de nodos no visitados). Además,
utiliza una fase de perturbación para diversificar la búsqueda. En pruebas con 67 instancias benchmark, MVNS obtuvo
65 de las mejores soluciones conocidas, demostrando ser robusto y eficiente en comparación con otros métodos como ACO-OP, VNS y NACO-OP.

Por otro lado, \cite{6552660} aborda el Rich Profitable Tour Problem (RPTP). Los autores proponen un
algoritmo híbrido que combina Variable Neighborhood Search con Adaptive Large Neighborhood Search (VNS/ALNS). Este método
utiliza una fase de perturbación basada en el paradigma Ruin and Recreate, que destruye parcialmente la solución y luego la
repara mediante heurísticas adaptativas de inserción y eliminación. La fase de mejora emplea una búsqueda local que reinserta
clientes no visitados. Cuando se evalúa en instancias del Orienteering Problem with Time Windows (OPTW), el algoritmo VNS/ALNS
obtiene soluciones óptimas o casi óptimas en 57 de 58 instancias, superando en rendimiento a métodos como ILS, GRASP/ELS y GVNS.

\subsubsection{Métodos híbridos}
En el contexto de sistemas de información turística, \cite{article} propone un marco híbrido para resolver el Enhanced Profitable
Tour Problem (EPTP). Este enfoque combina técnicas de reducción de grafos, construcción inicial de rutas y mejora iterativa
mediante un algoritmo único de Extensión/Colapso. Primero, el grafo original de la red vial se reduce conservando solo los nodos de
interés y calculando caminos óptimos entre ellos mediante un algoritmo de Dijkstra modificado que considera preferencias del usuario.
Luego, se genera una solución inicial factible insertando nodos de manera iterativa según un criterio de ganancia por unidad de tiempo.
Finalmente, la fase de mejora alterna entre una Fase de Extensión, que añade nodos prometedores al tour, y una Fase de Colapso, que
elimina nodos para mantener la factibilidad respecto al tiempo máximo. Este enfoque híbrido logra soluciones de alta calidad
(más del 90\% del óptimo en comparación con branch and bound) en tiempos computacionales razonables (3.7 segundos para un tour de 8
horas), siendo adecuado para aplicaciones en tiempo real.

En un enfoque más reciente, \cite{HE2025107077} introduce el Capacitated Profitable Tour Problem with Cross-Docking (CPTPC),
un problema de enrutamiento de dos niveles donde empresas logísticas seleccionan solicitudes de transporte rentables desde plataformas
industriales y utilizan un cross-dock para consolidar envíos. Los autores desarrollan un Hybrid Genetic Algorithm (HGA) que incorpora un
operador de cruce de dos niveles EAX² para combinar rutas de recogida y entrega de soluciones parentales, manteniendo bloques prometedores.
El algoritmo también incluye una búsqueda local con once operadores de vecindad (M1-M11) y una técnica de cálculo optimizada que utiliza
colas de prioridad para evaluar inserciones de nodos de entrega de manera eficiente. Las pruebas en instancias reales de 36 solicitudes
muestran que HGA mejora los beneficios entre un 59.74\% y un 124.31\% respecto a métodos manuales, demostrando su utilidad práctica
para la logística basada en plataformas digitales.

\subsubsection{Técnicas emergentes}
Finalmente, \cite{9989933} aborda el Online Stochastic Profitable Tour Problem
(OS-PTP), una variante del TSP con beneficios donde las ganancias de los clientes se modelan mediante variables aleatorias dependientes
del tiempo cuyas realizaciones se revelan de forma incremental. Los autores proponen un algoritmo de Deep Reinforcement Learning (DRL)
basado en AlphaZero, que combina una red neuronal convolucional con una búsqueda en árbol de Monte Carlo (MCTS). La red neuronal
procesa matrices de entrada que representan la posición actual del agente, los beneficios esperados y el tiempo restante, generando
dos salidas: un valor de estado (value-head) que estima la ganancia futura esperada y una distribución de probabilidad (policy-head)
sobre los próximos clientes a visitar. Durante la fase en línea, el MCTS guiado por la red, explora iterativamente los movimientos más
prometedores mediante una función UCT que balancea exploración y explotación. El entrenamiento se realiza mediante auto-juego en
1,000 escenarios de cada instancia, actualizando la red con los datos generados por el MCTS. En pruebas con instancias de 10 a 50
clientes, el algoritmo mejora progresivamente su rendimiento, alcanzando hasta un 97\% del óptimo de referencia tras 4 días de
entrenamiento. Este enfoque representa un avance significativo en la aplicación de DRL a problemas de enrutamiento estocásticos,
eliminando la necesidad de diseñar heurísticas manuales y adaptándose dinámicamente a la incertidumbre en la demanda.

\subsection{Resultados obtenidos}
Si bien la revisión abarca una gama de problemas relacionados (PTP, OP, TOP), el análisis comparativo se orienta a identificar las
técnicas más adecuadas para abordar el Enhanced Profitable Tour Problem (EPTP), el cual se caracteriza por la selección óptima de
clientes, la maximización de la utilidad y, críticamente, la incorporación de restricciones operativas realistas como ventanas de tiempo.

Desde esta perspectiva, los métodos exactos \cite{LERAROMERO2021879} quedan descartados para instancias realistas de EPTP debido a su
alta complejidad computacional. La introducción de ventanas de tiempo y otros recursos hace que el espacio de búsqueda sea demasiado
grande para que estos métodos sean viables en aplicaciones prácticas, aunque siguen siendo valiosos para obtener soluciones de referencia
en instancias pequeñas.

Entre las metodologías aproximadas, las metaheurísticas y heurísticas avanzadas demuestran ser las más robustas. En particular,
los enfoques híbridos que combinan diferentes estrategias de búsqueda son los que mejor se adaptan a la naturaleza combinatoria y
fuertemente restringida del EPTP. El algoritmo MVNS de \cite{5668224} y el VNS/ALNS de \cite{6552660}, ambos probados en problemas
con ventanas de tiempo (OPTW), son ejemplos destacados. Su éxito radica en la capacidad de equilibrar una búsqueda local intensiva
(para explotar soluciones prometedoras) con mecanismos de diversificación efectivos, para escapar de óptimos locales.

Por otro lado, las técnicas emergentes como el DRL \cite{9989933} si bien son extremadamente prometedoras para entornos dinámicos y
estocásticos (una posible extensión futura del EPTP), actualmente presentan una barrera de entrada alta por su costo de entrenamiento
y complejidad de implementación. Para el EPTP en su formulación estática y determinística, las metaheurísticas híbridas consolidadas
ofrecen una mejor relación costo-beneficio.

\section{Modelo Matemático}
El siguiente modelo matematico fue creado usando \cite{article} y el documento entregado por el ayudante asociado a mi problema Marcel
Silva.

Dado un grafo dirigido $G = (V, A)$ donde $V$ es un conjunto de $n$ nodos, y $A$ es un conjunto de $m$ arcos, considerando solo
1 usuario por instancia:

\subsection{Parámetros y constantes}
\begin{itemize}
  \item $V = \{1, 2, . . . , n\}$: conjunto de nodos.
  \item $A \subseteq V$ x $V$: conjunto de arcos dirigidos.
  \item $t_i$: tiempo de servicio en el nodo $i$. $i \in V$.
  \item $d_{ij}$ : tiempo de viaje asociado al arco ($i$, $j$) $\in A$.
  \item $T$: tiempo total del usuario.
  \item $c_{ij}$: valorización del arco ($i$, $j$) que le da el usuario. $i$, $j \in V$
  \item $s_i$: valorización del nodo $i$ que le da el usuario. $i \in V$
  \item $[e_i, l_i]$: ventana de tiempo en la que el nodo $i$ está disponible. $i \in V$
\end{itemize}

\subsection{Variables}
\begin{itemize}
  \item $x_{ij} \in$ \{0, 1\}: variable que indica si el arco ($i$, $j$) es parte del tour.
  \item $y_i \in$ \{0, 1\}: variable que indica si el nodo $i$ es visitado.
\end{itemize}

\subsection{Función Objetivo}

\begin{equation}
maximize \sum_{i \in V} s_i y_i + \sum_{i \in V}\sum_{j \in V \setminus \{i\}} c_{ij} x_{ij}
\end{equation}

(1) Representa la función objetivo, expresa la maximización de las valorizaciones de los arcos y los nodos por parte
del usuario.

\subsection{Restricciones}
\begin{equation}
\sum_{j \in V \setminus \{i\}} x_{ij} - y_i = 0, \quad \forall i \in V
\end{equation}
\begin{equation}
\sum_{i \in V \setminus \{j\}} x_{ij} - y_j = 0, \quad \forall j \in V
\end{equation}
(2) y (3) se aseguran que los nodos solo tengan 1 arco de entrada y 1 arco de salida.

\begin{equation}
y_1 = 1
\end{equation}
En (4) por simplicidad se determina el nodo inicial fijo.

\begin{equation}
\sum_{i \in V}\sum_{j \in V \setminus \{i\}} d_{ij} x_{ij} + \sum_{i \in V} t_i y_i \leq T
\end{equation}
En (5) se asegura de que no se sobrepase el tiempo maximo del usuario.

\begin{equation}
\sum_{i \in S}\sum_{j \in V \setminus S} x_{ij} \geq y_h, \quad \forall S \subset V \quad : \quad 1 \in S, h \in V \setminus S
\end{equation}
La ecuacion (6) garantiza que el ciclo encontrado este conectado e incluya al nodo de inicio. La restricción asegura que si un nodo
$h$ está incluido en el ciclo (es decir, $y_h = 1$), entonces debe existir al menos un arco que salga del conjunto $S$ y entre a
$V \setminus S$. Esto conecta $h$ con el nodo 1, ya que $S$ siempre contiene al nodo 1. Junto con (2) y (3) se garantiza que solo exista
un ciclo. Este se llama ``sub-tour elimination constraint''.

\begin{equation}
x_{ij} \in \{0, 1\}, \quad \forall (i,j) \in A
\end{equation}
\begin{equation}
y_i \in \{0, 1\}, \quad \forall i \in V
\end{equation}
Finalmente en (7) y (8) se definen la naturaleza de las variables.

\subsection{Espacio de búsqueda}
La variable $y_i$ tiene espacio $2^n$, mientras que la variable $x_{ij}$ tiene espacio $2^{n^2}$,
juntando ambas, el espacio total de este modelo es $2^{n + n^2}$ 

\section{Conclusiones}
El análisis del estado del arte revela que, si bien existen diversas técnicas para problemas relacionados con el Enhanced
Profitable Tour Problem (EPTP), no todas abordan exactamente el mismo problema. Las metodologías examinadas presentan tanto similitudes
como diferencias fundamentales en su enfoque.

Las técnicas existentes se asemejan en su objetivo común de maximizar la rentabilidad en problemas de enrutamiento selectivo, y en
la necesidad de balancear la exploración y explotación del espacio de búsqueda. Particularmente, métodos como VNS/ALNS \cite{6552660}
y el enfoque híbrido de \cite{article} comparten estrategias similares de búsqueda local adaptativa y mecanismos para mantener la
factibilidad frente a restricciones operativas.

También algoritmos como MVNS \cite{5668224} demuestran eficacia en problemas más simples como el OP, su adaptación al EPTP requeriría
modificaciones sustanciales para incorporar la valorización de arcos y las complejas restricciones temporales.

Entre las limitaciones identificadas destacan: la escalabilidad reducida de métodos exactos \cite{LERAROMERO2021879}, la dependencia
de parámetros en metaheurísticas tradicionales, y el alto costo computacional de técnicas emergentes como DRL \cite{9989933}.

Como trabajo futuro interesante, se propone desarrollar un algoritmo híbrido que combine la adaptabilidad de ALNS con mecanismos de
aprendizaje automático para la selección automática de operadores. Específicamente, se sugiere incorporar:
\begin{itemize}
\item Estrategias de búsqueda adaptativa basadas en el estado actual de la solución
\item Técnicas de poda inteligente que aprovechen la estructura del problema
\end{itemize}

Estas contribuciones potenciales podrían mejorar significativamente la resolución del EPTP al reducir los tiempos de convergencia y
aumentar la calidad de las soluciones en instancias de gran escala, particularmente en contextos logísticos urbanos donde las
restricciones temporales son críticas.

\bibliographystyle{plain}
\bibliography{Referencias}

\end{document} 
